\documentclass[a4paper,10pt]{report}
\usepackage[utf8]{inputenc}
\usepackage{algorithm}
\usepackage{algorithmic}
\usepackage{hyperref}

% Title Page
\title{Trifocal Tensor}
\author{Cédric Bidaud & Aurélien Greffard}


\begin{document}

\maketitle

\section*{Project state sum-up}

\begin{itemize}
 \item Asked working implemented elements :
 \begin{itemize}
   \item[-] help
   \item[-] options in command line
   \item[-] points lists management
   \item[-] images display
   \item[-] image loading from command line
   \item[-] list loading from command line
   \item[-] calculation of the tensor
   \item[-] transfer (works, but the approximation is not negligible)
 \end{itemize}
 \item Asked but not working implemented elements :
 \begin{itemize}
  \item[-] Transfer optimisation. The points given by the transfer function are in the 
  right zone, often very close to their theorical location, but sometimes very far from it.
 \end{itemize}
 \item Asked but not implemented elements : none
 \item Not (explicitly) asked and workink implemented elements : error messages
 \item Not asked and not working (or not implemented) elements : none
\end{itemize}

Report written in LaTeX.


\chapter{Analysis}
\section{Tensor calculation}


From the equation extracted and given in the subject :
\[x^{k}(x'^{i}x''^{3}T^{3l}_{k} - x'^{3}x''^{3}T^{il}_{k} - x'^{i}x''^{l}T^{33}_{k} + x'^{3}x''^{l}T^{i3}_{k}) = 0^{il}  (1)\]
which actually corresponds to a system of several equations. In order to calculate the vector t, we have to rewrite this system as 
a matrix of type \begin{math}At = 0\end{math}, which means to build this matrix A. Each element of A corresponds to the coefficients in the equations 
which are just before elements of t. Different loops will allow us to insert correct coefficients in their right places in the matrix.
\\

First step consists in initializing elements of A to 0. Indeed, in each equation of the system, only 12 elements of t on 27 appear, which 
means others have a null coefficient. Eigen provides some tools to directly initialize matrix A to 0.
\\

Second step consists in filling matrix A. When examining equation (1), we understand this filling will be managed with 4 loops : 3 loops 
on p, i and l, which select the right line of the matrix, because they are variables which create new equations, then a loop on k, which
 selects the right column, according to the t coordinate we are working on.
\\

In the subject, i and l vary from 1 to 2, p from 1 to the number of match points, and k from 1 to 3. As lines and columns of our 
matrices start from 0, we will have i and l varying from 0 to 1, k from 0 to 2 and p from 0 to the number of points -1.
\\

Coefficients we will insert in A will depend on coordinates of x, x’ et x” points. Now, these coordinates are saved in our matching points
 lists, which means in a matrix form. We have to be able to get them back too. For instance, if we consider that points of the first, 
second and third image are respectively saved in matrices list1, list2 and list3, then k coordinates of the matching point p of the 
first image is list1(p, k).
\\

Last step consists in calculating t with SVD method, provided by Eigen. Matrix A is decomposed in a 3 matrices product U*D*V, and we can save the tensor values, which correspond to the last column of V (column 26 in our case).
\\

This way, still from equation (1), we get to the following pseudocode :
\begin{algorithm}
\caption{Tensor calculation}
\begin{algorithmic}
\STATE Initiate A to 0
\FOR{$p=0$ to $nbPoints - 1$}
\FOR{$i=0$ to $1$}
\FOR{$l=0$ to $1$}
\FOR{$k=0$ to $2$}
\STATE A(4*p + 2*i + l, 9*2 + 3*l + k) += list1(p,k) * list2(p,i) * list3(p,2)
\STATE A(4*p + 2*i + l, 9*i + 3*l + k) -= list1(p,k) * list2(p,2) * list3(p,2)
\STATE A(4*p + 2*i + l, 9*2 + 3*2 + k) -= list1(p,k) * list2(p,i) * list3(p,l)
\STATE A(4*p + 2*i + l, 9*i + 3*2 + k) += list1(p,k) * list2(p,2) * list3(p,l)
\ENDFOR
\ENDFOR
\ENDFOR
\ENDFOR
\STATE A = UDV \COMMENT{SVD decomposition}
\STATE $tensor \leftarrow V.col(26)$
\end{algorithmic}
\end{algorithm}

\section{Transfer}


Now that t is known, we want to find a point from the two others. The principle is the same : first rewrite the system as a matrix, and then fill a matrix B.
\\

Before starting and calculating a 3 coordinates vector just like in the last step, it could be a good thing to notice that the 3\textsuperscript{rd} 
coordinate of the point we are looking for is known and is equal to 1, as it is its homogeneous coordinate. We only have 2 unknown coordinates then 
and this time, we have to solve a matrix equation of the form \begin{math}Bv = b\end{math},  with v the 2 coordinates vector.
\\

We should rewrite equation (1) with this last point in mind. To be able to find a point on any image from the two others, we have to 
examine this equation for each of the 3 cases.
\\

If we are looking for the point x’, then x’3 is equal to 1, which leads us to the following equation :
\[x^{k}x'^{i}x''^{3}T^{3l}_{k} - x^{k}x'^{i}x''^{l}T^{33}_{k} = x^{k}x''^{3}T^{il}_{k} - x^{k}x''^{l}T^{i3}_{k}\]

If we are looking for the point x”, then x”3 is equal to 1, which leads us to the following equation :
\[x^{k}x'^{3}x''^{l}T^{i3}_{k} - x^{k}x'^{i}x''^{l}T^{33}_{k} = x^{k}x'^{3}T^{il}_{k} - x^{k}x'^{i}T^{3l}_{k}\]

Finally, if we are looking for the point x, then x3 is equal to 1, which leads us to the following equations :

if k=3 :
\[x'^{i}x''^{3}T^{3l}_{k} - x'^{3}x''^{3}T^{il}_{k} - x'^{i}x''^{l}T^{33}_{k} + x'^{3}x''^{l}T^{i3}_{k} = 0^{il}\]

else :
\[x^{k}(x'^{i}x''^{3}T^{3l}_{k} - x'^{3}x''^{3}T^{il}_{k} - x'^{i}x''^{l}T^{33}_{k} + x'^{3}x''^{l}T^{i3}_{k}) = 0^{il}\]

This time, we have to fill the matrix B and the vector b. To avoid confusions, we will use MatB and Vecb instead of B and b.

\begin{algorithm}
\caption{Transfer}
\begin{algorithmic}
\REQUIRE x1 and x2 are known
\STATE Initiate MatB to 0
\STATE Initiate Vecb to 0
\FOR{$i=0$ to $1$}
\FOR{$l=0$ to $1$}
\FOR{$k=0$ to $2$}
\IF{we search the point x}
\STATE factor = x1(i) * x2(2) * tensor(2, l, k) - x1(2) * x2(2) * tensor(i, l, k) - x1(i) * x2(l) * tensor(2, 2, k) + x1(2) * x2(l) * tensor(i, 2, k)
\IF{k=2}
\STATE Vecb(2*i + l) -= factor
\ELSE
\STATE MatB(2*i + l, k) += factor
\ENDIF
\ELSIF{we search the point x'}
\STATE MatB(2*i + l, i) += x1(k) * ( x2(2) * tensor(2, l, k) - x2(i) * tensor(2, 2, k) )
\STATE Vecb(2*i + l) -= x1(k) * ( x2(l) * tensor(i, 2, k) - x2(2) * tensor(i, l, k) )
\ELSIF{we search the point x''}
\STATE MatB(2*i + l, l) += x1(k) * ( x2(2) * tensor(i, 2, k) - x2(i) * tensor(2, 2, k) )
\STATE Vecb(2*i + l) -= x1(k) * ( x2(i) * tensor(2, l, k) - x2(2) * tensor(i, l, k) )
\ENDIF
\ENDFOR
\ENDFOR
\ENDFOR
\STATE MatB = UDV \COMMENT{SVD decomposition}
\STATE $solution \leftarrow SVD.solve(Vecb)$
\end{algorithmic}
\end{algorithm}


\chapter{The program}

In this part, we will explain our objects, structures and functionalities choices.
\\

\section{Data structures}
\subsection{Tensor}

The tensor is an object with a vector of n elements - Eigen::VectorXf - initialized with zeros
and methods to easily acces an element - operator () surcharge, and of course fill it and get its elements.
Once filled it can also calculate the transfered point from two points clicked on images.
\\

The fill and transfer methods could have been part of the Tensor class, but it was clearer
for us to set them apart.
\\

\subsection{Point lists}

The matching points are saved in lists. The first two numbers are the x and y coordinates
clicked on the image, the third is the homogeneous coordinate (1). To remain coherent,
the points are saved in the same order as the one they are clicked on the images.

In the program, lists are matrices - Eigen MatrixXf. For each matrix corresponds an int that 
saves its rows count. It's usefull to add a point to a list : as we strangely didn't find
any push method for Eigen matrices, we wrote ours. This implied to resize the matrix we wanted
to push, that's why we decided to save this number for a practical reason - also usefull for some tests.
\\

\subsection{Others}

To perform the transfer, we have to save the two clicked points and their corresponding images.
We define two sets : the firts for the points, in which we stock clicked points, the second initialized with
the three images. When an image is clicked, it is removed from this set, and the one which remains
is the image where the transfered point must be written.
\\

\section{Algorithms}
As algorithms used to calculate the tensor and the transfer function have already been explained
in the first chapter, we will here focus on algorithms related to the good working of the program.
\\

\subsection{Options}
One of the key features asked is the possibility to handle options entered in command line. As we never 
managed this kind of options, we had to find a solution to efficiently analyse arguments given to the
program. We choose to convert argv arguments in an array of strings - std::string. It provided us the
usefull strcmp method, which allows to compare a string to another.

\begin{algorithm}
\caption{Options}
\begin{algorithmic}
\FOR{each argument}
\FOR{each option}
\IF{argument corresponds to an option}
\STATE execute the option
\ENDIF
\ENDFOR
\ENDFOR
\end{algorithmic}
\end{algorithm}

The list of the different options implemented can be found on the first page.

As the searches for lists or images arguments rely on the same algorithm, we will only describe the one
for the images. We are located at the ``Execute the option'' point.

\begin{algorithm}
\caption{Search for images}
\begin{algorithmic}
\REQUIRE $externalImages = 0$, the number of loaded images
\IF{argument corresponds to an option (contains .jpg, .png or .gif)}
\IF{$externalImages = 3$}
\STATE too many images loaded, the first three are kept
\ELSE 
\STATE load image (if possible) and increment $externalImages$
\ENDIF
\ENDIF
\end{algorithmic}
\end{algorithm}

This algorithm assures us that the user won't load too many images.
To be sure that he doesn't load less, we make another verification after browsing all the arguments :
if the number of external images is less than 3, the program loads default images, with a message
which indicates if there is no or not enough images loaded.

\subsection{Program states}
The program runs with three states :
\begin{description}
 \item[FILL\_LISTS] is the standard state, where the user fills the matching points lists by clicking on the images.
 If the three lists are filled, the tensor calculation can be launched by pressing Enter.
 \item[TRANSFERT] is the state the program enters if the tensor is successfully calculated. The user must now
 click two points on two different images to launch the calculation of the coordinates of the
 transfered point.
 \item[SOLUTION] is the last state, where the program displays the transfered point according to the two points clicked 
 in TRANSFERT state.
\end{description}

These three states alter the behaviour of input commands (mouse and keyboard) and are checked in the event management loop.


\section{Optimisations}

As we said in the sum-up at the begining of this report, the transfer function is not always accurate.
We have several ideas to improve the precision, but didn't manage or didn't have the time to implement them. For example : \url{http://users.cecs.anu.edu.au/~hartley/Papers/tensor/journal/final/tensor3.pdf} page 9 section Normalization.
We will quote here this report :
\begin{quotation}
  Normalization. Before setting out to write and solve the equa-
  tions, it is a very good idea to normalize the data by scaling and
  translating the points. The algorithm does not do well if all points
  are of the form (u1 , u2 , 1) in homogeneous coordinates with u1 and
  u2 very much larger than 1. A heuristic that works well is to trans-
  late the points in each image so that the centroid of all measured
  points is at the origin of the image coordinates, and then scaling so
  that the average distance of a point from the origin is 2 units. In
  this way the average point will be something like (1, 1, 1) in homo-
  geneous coordinates, and each of the homogeneous coordinates will
  be approximately of equal weight. This transformation improves the
  condition of the matrix of equations, and leads to a much better solu-
  tion. Despite the seemingly harmless nature of this transformation,
  this is an essential step in the algorithm.
\end{quotation}
\\

Another way to improve our program would be to set up a real errors manager. For the moment, the program just
displays an error message then crashes if a problem appears.

\end{document}